\chapter{Final Product Evaluation}
\label{ch:eval}
\section{Evaluation of Requirements}
In this section, it will be evaluated whether the requirements set in section \ref{sec:req} were met. This evaluation can be found in table \ref{table:requieva}.\\

{\renewcommand{\arraystretch}{1.5}
	\centering
	\begin{tabular}{ | l | l || l || }
		\hline
		\bfseries{Priority} & \bfseries{Requirement} & \bfseries{Met} \\ \hline
		M & Establish a wireless connection & Yes \\ \hline
		M & Send sensor data over a wireless connection & Yes \\ \hline
		M & Format data in a comprehensible way & Yes \\ \hline
		M & Access sensor data by CUI on external PC & Yes \\ \hline
		M & Work with custom Simulink block (API) & Yes \\ \hline
		S & Basic GUI for sensor data on external PC & Yes \\ \hline
		C & Sophisticated GUI for sensor data on external PC & Yes \\ \hline
		C & Graphical representation of sensor data in GUI & Yes \\ \hline
		C & Data logging on the external system & Yes \\ \hline
		W & Broadcast analysed data to multiple interfaces & Yes \\ \hline
		W & Represent the exoskeleton with a 3D-model in the GUI based on the data & No \\ \hline
		W & Send commands from external system to exoskeleton & No \\ \hline 
	\end{tabular}
	\captionof{table}{Evaluation of Requirements} 
	\label{table:requieva}
}

\section{Evaluation of Success Criteria} \label{sec:evasuc}
In section \ref{sec:suc} the criteria for when the project could be called `done' and when it could be called a success were defined.

In table \ref{table:requieva} it can be seen that all requirements with priorities `Must have' and `Should have' have been met. Furthermore, the  of the requirements with priority `Could have' have been met as well. The system also ran without issues on the hardware of the exoskeleton. Therefore the project can be called a success.