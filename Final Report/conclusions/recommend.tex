\chapter{Recommendations}
\label{ch:rec}
\section{3D-Model}
In the original description of the project, creating a 3D-Model of the MARCH based on the sensor data was one of the main requirements of the system. However, after research it turned out that the system which was previously used to send the wireless data did not work, this requirement was pushed back in order to create a working data transfer system.  

Another member of the Project MARCH team has created a simple 3D-model of the MARCH which can be controlled using a Simulink connection. Connecting this model to the data receiver is would be advised compared to trying to integrate this model in the GUI, as this will likely cause performance problems.

\section{Multiple windows}
When the system is used in practice, the members of Project MARCH want to have multiple screens with the sensor data. Since the data is broadcast over the wireless network, it is possible to receive the data on multiple laptops which each can be set to show different sensors. In case that multiple monitors must be used by the same laptop, the GUI can be modified to generate multiple windows, or one windows spanning multiple fields. This will have a negative impact on the update rate though.

\section{Emergency Stop}
If a sensor indicates that the MARCH is malfunctioning, it is desirable that the exoskeleton is put to a standstill immediately. While there is a physical emergency stop on the MARCH, the Project MARCH team should be able to shut it down remotely the moment an abnormality is detected. While this is outside the scope of this project, the wireless system can be used to send a packet from the GUI with a shut-down signal to the MARCH. 

\section{Retrieve properties from sensor type}
The MARCH contains many sensors of which some are the same type. These sensors have mostly the same properties and therefore the default values of a sensor could be retrieved from a configuration file based on the sensor type. This would reduce the error of manually setting the properties of each sensor.

\section{More GUI elements}
The GUI can easily be expanding with more elements, these include but are certainly not limited to the following:

\begin{itemize}
	\item Showing how long the GUI has been running
	\item Using other kind of graphs than only line graphs
	\item Showing the mean, deviation and other statistical information about the values of a sensor
\end{itemize}

\section{Recording Feedback}
The recording toggle in the GUI is currently based on use on one computer at a time. It does not account for multiple GUIs being active at the same time or a crashing GUI. The recording state is being send in the footer of every packet, however it is not being used yet. An extension of the functionality would be to set the recording toggle accordingly to the status in the footer.

%GG Java engine ombouwen