\chapter{Introduction}
%AA Short Intro Project March
The team of project MARCH consists of a young multidisciplinary group of students from Delft University of Technology, with one vision: to re-enable paraplegics to participate in daily life activities. In order to realise this Project MARCH is designing and building a powered exoskeleton. One of the goals of Project MARCH is to participate in the ‘Powered Exoskeleton Race’-discipline of the 2016 Cybathlon in Zurich to show their progress with this technology. The Cybathlon is the first ‘Olympics’ for disabled athletes using bionic assistive technology.
 
%BB What Project March Needs us for
The exoskeleton has many sensors which acquire data from their surroundings and are further processed in Simulink. Project MARCH requires a system which allows monitoring and visualization of all vital sensor data in order to detect abnormalities and prevent damage to the pilot or the MARCH. Since having a cable from the MARCH to a stationary site would impend the movement, the monitoring has to be wireless. 		

%DD How we implemented it			
To create a wireless monitoring system, several parts have to be build in order for the total system to properly work. These parts are: a system to extract the data from all the relevant sensors, a system to send this data wireless and a system to receive and show the data.

\section{Structure of this report}
In chapter \ref{ch:res} the research process is described together with the found conclusions. In chapter \ref{ch:imp} the implementation process is documented with how the product was implemented as well as the found difficulties. In chapter \ref{ch:eval} it is evaluated whether the requirements were met. In chapter \ref{ch:rec} several recommendations are done for possible future features of the products.  In chapter \ref{ch:con}, the conclusion about the final product is given.
%AA Laten liggen tot hoofdstukindeling zekerder is /not sure if nodig(?)