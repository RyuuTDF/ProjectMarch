\section{Problem definition \& analysis} \label{sec:probdef}
The first thing to be done was to place the problem in the context of the project. In section \ref{sec:prodef} the situation of the client that led to the problem is described, then in \ref{sec:proana} an analysis is done of why the problem is an issue and should be solved.

\subsection{Problem definition}\label{sec:prodef}
In 2015, the ``Moving Bird Foundation'' was founded. As part of the foundation, the dream team ``Project MARCH'' was created. The goal of this dream team is designing and building an exoskeleton for paraplegic people, to give them back the ability to stand up and walk, to overcome obstacles they face in daily life and to be able to communicate on eye level again. The dream team will participate in the `Powered Exoskeleton Race'-discipline of the 2016 Cybathlon in Zürich\footnote{\url{http://www.cybathlon.ethz.ch/en/}}. The Cybathlon is the first `Olympics' for disabled athletes using bionic assistive technology.
 
The exoskeleton that is being designed has many sensors, which acquire data from their surroundings. Project MARCH wishes to have a system, which allows for external monitoring and visualization of all vital sensor data.

\subsection{Problem Analysis}\label{sec:proana}
During this project the team is tasked to design and implement an external system, that hooks into the existing system and makes the monitoring of the sensors possible. Currently there is no wireless connection, the only way to check the sensor data is by connecting the exoskeleton to an external PC with a LAN-cable. This way of connecting is not preferred, as it has the possibility to impede the movement of the exoskeleton.

Furthermore, in the current state only the raw sensor data can be extracted, which is of no use since this data is unitless. It is important for the MARCH-team to be able to monitor the sensors in a comprehensible way, as they will be able to prevent problems, if failing parts of the exoskeleton are spotted faster.