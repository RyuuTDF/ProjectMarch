\section{Requirements}\label{sec:req}
Since the problem definition and analysis were concrete enough, it was possible to create a list of requirements directly from them. After meeting with the Client Advisor to discuss this list, the requirements were updated to the current needs and can be found in table \ref{table:requi}. The result will be taken in consideration in the rest of the report.\\

{\renewcommand{\arraystretch}{1.5}
	\centering
	\begin{tabular}{ | l | l | }
		\hline
		\bfseries{Priority} & \bfseries{Requirement} \\ \hline
		M & Establish a wireless connection \\ \hline
		M & Send sensor data over a wireless connection \\ \hline
		M & Format data in a comprehensible way \\ \hline
		M & Access sensor data by CUI on external PC \\ \hline
		M & Work with custom Simulink block (API) \\ \hline
		S & Basic GUI for sensor data on external PC \\ \hline
		C & Sophisticated GUI for sensor data on external PC\\ \hline
		C & Graphical representation of sensor data in GUI\\ \hline
		C & Data logging on the external system \\ \hline
		W & Represent the exoskeleton with a 3D-model in the GUI based on the data \\ \hline
		W & Broadcast analysed data to multiple interfaces \\ \hline 
		W & Send commands from external system to exoskeleton \\ \hline 
	\end{tabular}
	\captionof{table}{Priorities ordered with the MoSCoW method} 
	\label{table:requi}
}

\section{Success Criteria} \label{sec:suc}
For a software development project, it is important to set criteria for when a project can be called done and when it can be called a success. The team deemed the project sufficiently completed if all requirements with the priorities `Must have', `Should have' would be met. If the majority of the requirements with priority `Could have' would be met too, the project could be called success.
