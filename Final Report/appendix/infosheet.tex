\chapter{Infosheet}
\pagebreak
\begin{small}
\subsubsection{General Information}
{\bfseries Project Title:} Exoskeleton Monitoring System\\
{\bfseries Client Organisation:} Project March\\
{\bfseries Presentation Date:} 24 June 2016

\subsubsection{Description}
The goal of Project MARCH is designing and building an exoskeleton for paraplegic people and participate with it in the 2016 Cybathlon in Zürich. The exoskeleton has many sensors, which acquire data from their surroundings. Project MARCH wanted to have a system, which allows for wireless external monitoring of the exoskeleton. 

During the research phase, we analysed the various limitations of the situation and searched for an implementation that worked around those. We encountered a couple of challenges during the project, the biggest one for each subsystem was Simulink's limitation on code generation, wireless packet loss and the high update rate caused issues with some of Matlab's GUI elements.  

As the client was building an exoskeleton that had to meet several rules, the base on which we were building was robust. Therefore the requirements did not change during the implementation phase. The implementation could be split up in three parts, so each team member was responsible for one subsystem. We choose Scrum as our methodology to still allow flexibility if something came up.

The product created exists of three parts: a Simulink library, software that runs on a Raspberry Pi and a Matlab GUI. We tested them using manual tests. We managed to implement all the requirements that were not labeled 'Would have'. 

We implemented all the core features the client wanted, however it is certainly possible to extend the GUI with more features that would enhance the system.

\subsubsection{Team Roles}
{\bfseries Name:} Jens Voortman\\
{\bfseries Interests:} Algorithms, Automata theory \& Video Editing\\
{\bfseries Contributions \& Role:} Document Master, GUI Back end, Simulink Design\\\\
{\bfseries Name:} Ruben Visser\\
{\bfseries Interests:} Networks, Web Development\\
{\bfseries Contributions \& Role:} Wireless Transfer Design\\\\
{\bfseries Name:} Vasco de Bruijn\\
{\bfseries Interests:} Complexity Theory, Interaction Design\\
{\bfseries Contributions \& Role:} GUI Design\\\\
All team members contributed to the reports and the final presentation.

\begin{wrapfigure}{r}{0.25\textwidth}
	\raggedleft
	\includegraphics[scale=0.0315]{logo/Project_MARCH2}
\end{wrapfigure}

\subsubsection{Client \& Coach}
{\bfseries Client:} N. Tsutsunava, Chief Engineer, Project March\\
{\bfseries Coach:} H.K.M. Huijgens, SERG group, TU Delft

\subsubsection{Contacts}
\begin{tabular}{l l}
	Jens Voortman & jensvoortman@outlook.com\\
	Ruben Visser & r.visser@lunoct.nl\\
	Vasco de Bruijn & vascodebruijn@gmail.com\\
\end{tabular}\\\\
The final report for this project can be found at: \url{http://repository.tudelft.nl}
\end{small}