\chapter{SIG Evaluation}
\pagebreak
\section{First Submission}
\subsection{Feedback}
De code van het systeem scoort 4 sterren op ons onderhoudbaarheidsmodel, wat betekent dat de code bovengemiddeld onderhoudbaar is. De hoogste score is niet behaald door een lagere score voor Unit Size en Unit Complexity.\\\\
Voor Unit Size wordt er gekeken naar het percentage code dat bovengemiddeld lang is. Het opsplitsen van dit soort methodes in kleinere stukken zorgt ervoor dat elk onderdeel makkelijker te begrijpen, te testen en daardoor eenvoudiger te onderhouden wordt. Binnen de langere methodes in dit systeem, zoals bijvoorbeeld de ‘forwarder.main’-methode, zijn aparte stukken functionaliteit te vinden welke ge-refactored kunnen worden naar aparte methodes. Commentaarregels zoals bijvoorbeeld '//Setup socket for sending data' en '//Receive loop' zijn een goede indicatie dat er een autonoom stuk functionaliteit te ontdekken is. Het is aan te raden kritisch te kijken naar de langere methodes binnen dit systeem en deze waar mogelijk op te splitsen.\\\\
Voor Unit Complexity wordt er gekeken naar het percentage code dat bovengemiddeld complex is. Ook hier geldt dat het opsplitsen van dit soort methodes in kleinere stukken ervoor zorgt dat elk onderdeel makkelijker te begrijpen, makkelijker te testen en daardoor eenvoudiger te onderhouden wordt. In dit geval komen de meest complexe methoden ook naar voren als de langste methoden, waardoor het oplossen van het eerste probleem ook dit probleem zal verhelpen.\\\\
Als laatste nog de opmerking dat er geen unit test-code is gevonden in de code-upload. Het is sterk aan te raden om in ieder geval voor de belangrijkste delen van de functionaliteit automatische tests gedefinieerd te hebben om ervoor te zorgen dat eventuele aanpassingen niet voor ongewenst gedrag zorgen.\\\\
Over het algemeen scoort de code bovengemiddeld, hopelijk lukt het om dit niveau te behouden tijdens de rest van de ontwikkelfase.
\subsection{Improvements}
To address the feedback of the SIG, all our code was thoroughly reviewed on complex methods. Most of the programs written in C were refactored to reduce complexity, function length and duplication of code in different programs. Also our Matlab code was adapted to reduce function size and complexity.
The feedback also mentioned that there was no test code present in the uploaded code and recommended to write automatic tests for the most important functionalities. Unfortunately, this was not possible as the most important functionalities deal with either network processes and communication or visual rendering. Neither of which can be properly tested automatically.

\section{Second Submission}
In de tweede upload zien we dat zowel de omvang van het systeem is gestegen, terwijl de score voor onderhoudbaarheid ongeveer gelijk is gebleven. 
De score voor Unit Size en Unit Complexity is nu licht verbeterd. Echter blijven deze eigenschappen nog steeds kritische punten. Er zijn nog steeds lange en complexe units in de code aanwezig. Een goed voorbeeld is de unit ‘main’ in het ‘convert.c’ bestand, waar aparte stukken functionaliteiten te vinden zijn. 
Verder blijft de opmerking dat er geen testcode in het systeem is toegevoegd. 
Uit deze observaties kunnen we concluderen dat de aanbevelingen van de vorige evaluatie nauwelijks zijn meegenomen in het ontwikkeltraject.