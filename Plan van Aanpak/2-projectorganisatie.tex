\section{Projectorganisatie}
Bij dit project zijn de volgende partijen betrokken:

\subsection{Project MARCH}
Project MARCH is een project om een exoskelect te ontwikkelen voor mensen met dwarslaesie zodat deze weer kunnen lopen. Voor dit project is Project MARCH de opdrachtgever en heeft het team de opdracht gegeven om een monitoring systeem te bouwen voor het exoskelet

\subsubsection{Client Advisor}
De Client Advisor is een contactpersoon vanuit Project MARCH die de kwaliteit en de doelen van Project MARCH te waarborgen. In dit geval is dat Nick Tsutsunava, de Chief Engineer van Project MARCH. Als Chief Engineer levert Nick ook de technische vereisten en kennis.

\subsection{TU Coach}
De TU Coach waarborgt de educatieve aspecten van het project en ondersteunt het team procesmatig in het project. In dit geval is de TU Coach Hennie Huijgens. 

\subsection{BEP-Coordinator}
De BEP-Coordinator controleert of het project voldoet aan de gestelde vereisten vanuit de TU en geeft aan het einde van het project samen met de TU Coach en de Client Advisor een eindcijfer voor het project. De BEP-Coordinator van dit project is Felienne Hermans. 

\subsection{SIG}
De SIG (Software Improvement Group) geeft een oordeel en feedback over de geproduceerde code voor het project. 

\subsection{Team}
Het team bestaat uit drie studenten die een Bachelor Project uitvoeren. Dit team heeft als taak om de opdracht van Project March uit te voeren om een wijze in lijn met de kennis en kunde die wordt verwacht van een Bachelor-afstudeerder.