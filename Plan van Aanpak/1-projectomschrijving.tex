\section{Projectomschrijving}
Het dream team ``Project MARCH'' ontwerpt en bouwt naar aanloop voor de Cybathlon 2016 in Zurich een exoskelet voor dwarslaesiepatiënten. Dit exoskelet bevat vele sensoren die data verzamelen over hun omgeving en worden verwerkt in een model opgezet in Simulink.

Ons team gaat zich richten op het bouwen van een systeem dat deze sensordata kan monitoren en visualiseren. Dit losstaande systeem zal door gebruik te maken van draadloze communicatie inhaken op het bestaande systeem en de mogelijkheid bieden om deze monitoring en visualisatie in een interface te tonen zonder met kabels verbonden te zijn met het exoskelet.

De te ontwikkelen applicatie gaat een intuïtieve en gebruiksvriendelijke interface bieden om ruwe sensordata te kunnen weergeven en visualiseren. Daarnaast kan de applicatie de locatie en oriëntatie van verschillende onderdelen van het exoskelet ten opzichte van elkaar bepalen in 3D-ruimte en dat grafisch weergeven in de interface.
\subsection{Hoofddoelen}
\begin{enumerate}
 \item Een draadloze verbinding tussen het exoskelet en een extern systeem
 \item Sensordata vanaf het exoskelet kunnen verzenden naar het externe systeem
 \item Een eenvoudige GUI om de sensordata op het externe systeem te weergeven
\end{enumerate}
\subsection{Streefdoelen}
\begin{enumerate}
 \item Grafische weergave van sensordata in de opgezette GUI
 \item Grafisch overzicht van het exoskelet in de GUI op basis van de ontvangen sensordata
\end{enumerate}

