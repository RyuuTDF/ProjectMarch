\section{Development Methodology}
\subsection{Scrum}
Voor de tijdsindeling voor het project zullen we gebruik gaan maken van een methode gelijkend aan Scrum. Hiermee kunnen we in vier tweewekelijkse sprints periodiek deelproducten opleveren en is het mogelijk om een korte terugkoppelingslus aan te houden. Tevens is het mogelijk om wanneer nodig de planning bij te werken aan de hand van de voortgang van de te behalen doelen. 

\subsection{MoSCoW}
Voor het vaststellen van de prioriteiten van verschillende onderdelen van het te bouwen van het product, zullen we gebruik gaan maken van het MoSCoW model. Hiermee worden de onderdelen van het project gelabeld met de prioriteiten zoals vermeld in tabel \ref{table:moscow}. Door onze focus te richten op de onderdelen met een must have en should have prioriteit zullen de meest gewenste onderdelen als eerste gebouwd en opgeleverd worden.\\[1cm]


{\renewcommand{\arraystretch}{1.5}
	\centering
	
	\begin{tabular}{ | l | l | l | }
		\multicolumn{3}{c}{\bfseries{MoSCoW Method}} \\ \hline
		M & Must have & Essential for the system\\ \hline
		S & Should have & Important yet not essential for the system \\ \hline
		C & Could have & Features to be implemented if time permits \\ \hline
		W & Would have & Might be interesting to implement in the future \\ \hline 
	\end{tabular}
	\captionof{table}{Priorities MoSCoW method} 
	\label{table:moscow}
	
}
