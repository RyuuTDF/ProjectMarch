\section{Development Methodology}\label{sec:devmet}

\subsection{Scrum}
During the project Scrum will be used for scheduling and planning. By employing this method, four sprints of two weeks each will be held in which deliverables will be yielded and make sure a short feedback loop can be used. In addition, it will be possible to adjust the schedule according to progress made on the goals to be achieved.

The team at Project MARCH is already using scrum as their planning method, and the schedule of this project is adapted to fit in the Scrum planning of the software engineering team. Advantages hereof are integration with the team, the possibility to use existing communication channels and being within the feedback loop of the team.

\subsection{MoSCoW}
To assign priorities to various parts of the product that are going to be delivered, the MoSCoW model will be used. With this the parts of the project will be labelled with the priorities as mentioned in the table \ref{table:moscow}. By focussing on parts with `must have' and `should have' priorities, the most important parts will be made and delivered first.\\[1cm]

{\renewcommand{\arraystretch}{1.5}
	\centering
	
	\begin{tabular}{ | l | l | l | }
		\multicolumn{3}{c}{\bfseries{MoSCoW Method}} \\ \hline
		M & Must have & Essential for the system\\ \hline
		S & Should have & Important yet not essential for the system \\ \hline
		C & Could have & Features to be implemented if time permits \\ \hline
		W & Would have & Might be interesting to implement in the future \\ \hline 
	\end{tabular}
	\captionof{table}{Priorities MoSCoW method} 
	\label{table:moscow}
	
}
